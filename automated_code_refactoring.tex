\documentclass[conference]{IEEEtran}
\IEEEoverridecommandlockouts
% The preceding line is only needed to identify funding in the first footnote. If that is unneeded, please comment it out.
\usepackage{cite}
\usepackage{amsmath,amssymb,amsfonts}
\usepackage{algorithmic}
\usepackage{graphicx}
\usepackage{textcomp}
\usepackage{xcolor}
\def\BibTeX{{\rm B\kern-.05em{\sc i\kern-.025em b}\kern-.08em
    T\kern-.1667em\lower.7ex\hbox{E}\kern-.125emX}}
\begin{document}

\title{AI Assisted/Automated code refactoring\\
{\footnotesize How the development of AI may impact the future of code refactoring}
}

\author{\IEEEauthorblockN{Loris Tomassetti}
    \IEEEauthorblockA{Linz, Austria \\
        loris.tomassetti@outlook.com}
    \and
    \IEEEauthorblockN{Alexander Weißenböck}
    \IEEEauthorblockA{Linz, Austria \\
        alewei934@gmail.com}
}

\maketitle

\begin{abstract}
    This Paper aims to shed light onto developments in AI Assisted/Automated Code refactoring, how it can help the indtustry and which models work most efficiently to tackle different challenges code refactoring brings. This will be done by going over various literature describing first the challenges at hand and afterwards discussing several possible solutions that have been tested to gain a greater understanding and to generate an informed outlook into further developments of this technology.
\end{abstract}

\begin{IEEEkeywords}
    machine learning algorithms, software code refactoring, deep neural network
\end{IEEEkeywords}

\section{Introduction}
According to Fowler \cite{fowler2018refactoring}, refactoring involves modifying a software system to enhance its internal structure without changing its external behaviour. Numerous empirical studies have demonstrated a direct link between refactoring activities and improvements in code quality metrics. This body of evidence underscores the critical importance of refactoring in software engineers' priorities. \cite{aniche2020effectiveness}.\\
However, deciding when and how to refactor can be challenging for developers. Refactoring in an early stage may cost too in regards to potential benefits, and refactoring too late may cause the refactor to be an even more significant time commitment \cite{kruchten2012technical}.\\
Tools have been in the hands of many developers to make this process more streamlined for years. Analytics tools such as PMD, ESLint, and Sonarqube can be integrated into various stages of a developer's workflow (within IDEs, during code reviews, or as comprehensive quality reports—to identify bugs and provide suggestions for enhancing code quality)\cite{aniche2020effectiveness}.\\
Taking a closer look at these tools, however, reveals that they commonly have a lot of false positives, forcing developers to double-check their results more often than not. In some cases, the detection strategies are based on hard thresholds of just a handful of metrics, such as lines of code in a file (e.g. PMD's famous "problematic" classification occurring once a method reaches 100 lines per default)\cite{aniche2020effectiveness}. These simplistic ways of detection cannot cover the complexity of modern systems.\\
Manually analyzing hundreds of metrics and figuring out which ones are the cause of technical dept is very hard and almost impossible for tool developers, which is where machine learning-based solutions come into play \cite{leitch2003maintainability}.\\
We will take a closer look at how exactly different models go about this task in section \ref{automation}


\section{Refactoring}
\subsection{Uses of Refactoring}
Fowler states refactoring describes changing the internal structure of code without changing its behaviour.\cite{fowler2018refactoring} 
Refactoring can lead to improvements in code in various aspects. By reducing the code's complexity while keeping its behaviour identical, it becomes more human-readable and, therefore, more maintainable since it is easier to expand on it in the future. \cite{kaur2016analysis}
Another great advantage of refactoring is the reduction of technical debt. This term refers to a metaphorical debt when subpar solutions get chosen to accelerate software production. While the chosen solutions might suffice they often do not take further developments into account and cause problems later. \cite{techdebt}
And of course, the overall code quality can be improved.
The main difficulty in code refactoring lies in finding places to optimize, otherwise called "bad smells". These bad smells can be found virtually anywhere in the source code. Refactoring can happen on different levels, like class-level refactoring describing the extraction of classes into one or more sub-classes, while variable-level refactoring could be something like executing an action inline instead of creating an unnecessary variable. \cite{aniche2020effectiveness} This makes it very difficult to locate these bad smells and as a consequence time consuming and expensive.

\section{Approaches for Automation}

\subsection{Large Language Models}
\subsubsection{GPT Model}
\subsubsection{Github co-pilot}
\subsubsection{Fauxpilot Client}

\subsection{Dedicated Models}
\subsubsection{DNNFFz}

\section{Benefits of AI-Powered Refactoring}\label{benefits}
Logically, automated refactoring comes with all the benefits dicussed in the prior chapter, namely more readable, higher quality code that is more maintainable, but more importantly it makes all this possible in a shorter amount of time. \cite{negara2012using}
Automating the refactoring process is not a new idea at all and numerous tools have been developed and modern IDE's already support some refactorings. However those are often small scale and need input on what to refactor. \cite{usedisuserefactor}
It could also be verified that autmated refactorings bring a statistically significant time improvement to nearly every type of refactoring. \cite{negara2013comparative} This of course means less cost in extension. The same study also found that currently automated refactoring is more often used for smaller chnages, since there the refactoring is less error prone. 
Using AI the possible usecases could be significantly expanded and lead to greater time savings.

\section{Challenges and Limitations}
\subsection{Over-reliance on Automation}
Especially when using applications like Chat GPT users often rely too much on the model understanding their problem and give insufficient and/or vague requests which have a lot of room for interpretation. This can then, unsurprisingly, lead to undesired results. \cite{alomar2024refactor}

\subsection{Potential for Unintended Consequences}
A big reason software developers often choose to not use atomated refactoring is the unpredictability of the result. 
"[…] If I cannot guess, I don't use the refactoring. I consider it not worth the trouble. […]" \cite{usedisuserefactor} Since refactoring can span multiple files it can be hard for a developer to check all the changes after an automatic refactoring has occured.
Especially generative AI solutions which do not only handle smell detection but also the generation of solutions have often found to provide mixed results, which can lead to the behavior of the code being changed. This also leads to reduced trust to those tools by developers. \cite{10.1145/3397481.3450656}

\subsection{Performance Concerns}
While big strides in the technology are being made, it is still a present issue that some automatic refactorings do produce errors especially when misused \cite{7833023} 
Especially when using AI based tools it is not always given that the behaviour of the code is indeed the same as before, making the refactoring itself faulty. Baqais and Alshayeb could find quite a lack in checking for bahavior in a multitude of research papers, leaving out a crucial step. \cite{baqais2020automatic}

\subsection{Trust and knowledge gaps}
Some developers stated, when using an AI refactoring tool, it would probably be useful to know how it works in the background to potentially get better results. \cite{10.1145/3397481.3450656}
This would then require more focus on learning those Models. Also in another study Developers did not use certain automated refactorings of IDE's because they did not know of their existence. \cite{negara2013comparative}
In both of those cases better education about already existing tools alone could help improve performance.


\section{Future Directions and Research Opportunites}
Due to the discussed limitations in section \ref{challenges}, we can derive two takeaways.
It is crucial that we continue to conduct more research to determine the accuracy of the refactored code produced by AI systems. This is not just a task for a few, but a collective effort that requires the expertise and dedication of all involved in the field of AI-assisted code refactoring.
Furthermore, it will be necessary to improve the communication between tools and developers and improve education about existing tools so software engineers can utilize them to their full potential.\\
Automated code refactoring is a familiar idea and shows accurate results. AI-assisted refactoring brings discoveries on when and, more importantly, why code has to be refactored. GPT models can already be used to great effect, properly refactoring found research opportunities. Random Forest models are great at discovering refactoring opportunities, especially when trained on specific datasets. As fully automated code refactoring is still faulty, more research has to be conducted to increase trustability and allow reliance on AI-assisted code refactoring.

\subsection{Personalized Code Refactoring Suggestions}
\section{Discussion}


\section{Methodology}

\section{Conclusion and Outlook}

%\nocite{*} % Makes every ref appear despite not being cited
\bibliographystyle{plain} % We choose the "plain" reference style
\bibliography{refs} % Entries are in the refs.bib file
\end{document}
