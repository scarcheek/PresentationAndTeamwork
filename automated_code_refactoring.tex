\documentclass[conference]{IEEEtran}
\IEEEoverridecommandlockouts
% The preceding line is only needed to identify funding in the first footnote. If that is unneeded, please comment it out.
\usepackage{cite}
\usepackage{amsmath,amssymb,amsfonts}
\usepackage{algorithmic}
\usepackage{graphicx}
\usepackage{textcomp}
\usepackage{xcolor}
\def\BibTeX{{\rm B\kern-.05em{\sc i\kern-.025em b}\kern-.08em
    T\kern-.1667em\lower.7ex\hbox{E}\kern-.125emX}}
\begin{document}

\title{AI Assisted/Automated code refactoring\\
{\footnotesize How the development of AI may impact the future of code refactoring}
}

\author{\IEEEauthorblockN{Loris Tomassetti}
    \IEEEauthorblockA{Linz, Austria \\
        loris.tomassetti@outlook.com}
    \and
    \IEEEauthorblockN{Alexander Weißenböck}
    \IEEEauthorblockA{Linz, Austria \\
        alewei934@gmail.com}
}

\maketitle

\begin{abstract}
    This Paper aims to shed light onto developments in AI Assisted/Automated Code refactoring, how it can help the indtustry and which models work most efficiently to tackle different challenges code refactoring brings. This will be done by going over various literature describing first the challenges at hand and afterwards discussing several possible solutions that have been tested to gain a greater understanding and to generate an informed outlook into further developments of this technology.
\end{abstract}

\begin{IEEEkeywords}
    machine learning algorithms, software code refactoring, deep neural network
\end{IEEEkeywords}

\section{Introduction}\label{introduction}
Refactoring, as defined by Fowler\cite{fowler2018refactoring}, is “the process of changing a software system in such a way that does not alter the external behavior of the code yet improves its internal structure”. More and more empirical studies have since established a positive correlation between refactoring operations and code quality metrics. All this evidence hints at refactoring being a high-priority concern for software engineers.\cite{aniche2020effectiveness}.\\
However, deciding when and how to refactor can prove to be a challenge for developers. Refactoring in an early stage may be cost too much for what youre getting out of it, and refactoring too late may cause the refactor to be an even bigger time comittment.\cite{kruchten2012technical}\\
Tools have been in the hands of many developers to make this process more streamlined for years now. Analytics tools to sniff out bugs or give hints on how to improve code quality such as PMD, ESLint, and Sonarqube can be integrated in different stages of a developers' workflow, e.g. inside IDEs, during code review or as an overall quality report. \cite{aniche2020effectiveness}\\
Taking a closer look at these tools, however, reveals that they commonly have a lot of false positives, making developers lose their confidence in them. Often, the detection strategies are based on hard threshholds of just a handfull of metrics, such as lines of code in a file (e.g. PMD's famous "problematic" classification occuring once a method reaches 100 lines per default). These simplistic ways of detection simply aren't able to capture the full complexity of modern software systems.\\
Manually analyzing hundreds of metrics and figuring out which ones are the cause of technical dept is very hard and almost impossible for tool-developers, which is where machine learning-based solutions come into play.\\
We will take a closer look at how exactly different ML-Models go about this task in section \ref{benefits}


\section{Methodology}
\input{sections/methodology.tex}

\section{Refactoring}\label{refactoring}
Fowler states refactoring describes "changing the internal structure of code without changing its behaviour" \cite{fowler2018refactoring}.
Refactoring can lead to improvements in code in various aspects. By reducing the code's complexity while keeping its behaviour identical, it becomes more human-readable and, therefore, more maintainable since it is easier to expand on it in the future \cite{kaur2016analysis}.
Another great advantage of refactoring is the reduction of technical debt. This term refers to a metaphorical debt when subpar solutions get chosen to accelerate software production. While the chosen solutions might suffice they often do not take further developments into account and cause problems later \cite{techdebt}.
And of course, the overall code quality can be improved.
The main difficulty in code refactoring lies in finding places to optimize, otherwise called "bad smells". These bad smells can be found virtually anywhere in the source code. Refactoring can happen on different levels, like class-level refactoring describing the extraction of classes into one or more sub-classes, while variable-level refactoring could be something like executing an action inline instead of creating an unnecessary variable \cite{aniche2020effectiveness}. This makes it very difficult to locate these bad smells and as a consequence time consuming and expensive.

\section{Approaches for Automation}\label{automation}
This section will cover the different approaches for automation based on machine learning algorithms and will take a closer look at large language models used by thousands of software developers today. This paper will look at different metrics for each model but focus on three questions as far as data exists in these fields:
\begin{itemize}
    \item How well can it detect refactoring opportunities? While also taking a closer look at false positives and false negatives.
    \item How well does the model refactor a given source code? 
    \item How well can it detect refactoring opportunities and follow through with a potential refactor? Combining the two factors to determine how well it would do in real-world software development workflows.
\end{itemize}
\subsection{Large Language Models}
In the fields of ai assisted workflows, large language models (LLMs) have shifted from a niece speciality to an almost omnipotent tool which finds applications from image creation to code generation.\cite{meyer2024ai} LLMs are huge deep-learning models pre-trained on enormous amounts of data. They are especially known for their ability to be trained on datatest of specific domains but can also be used on a broad spectrum of general knowledge, making them incredibly flexible. \cite{baumgartner2024aidriven}
\subsubsection{GPT Model}
This section covers the experiments' results of the paper \cite[AI-Driven Refactoring: A Pipeline for Identifying and Correcting Data Clumps in Git Repositories]{baumgartner2024aidriven}.
Being the best-known LLM, OpenAI's Generative Pre-trained Transformer series (GPT) with its versions GPT-3.5, GPT-3.5 Turbo and GPT-4.
Temperature is a key parameter for GPT models, having a value ranging from zero to one, this parameter determines the predictability of the results. 
A higher temperature leads to more variety in the LLM and vice versa.\\
Taking a look at detection itself, the median sensitivity of GPT-3.5-Turbo is 0 which indicates many data clumps are undetected and many false positives. 
Submitting all files in bulk also made the model trade off its sensitivity with the specificity parameters, with the median sensitivity reaching 50\%, but the specificity only being 14\%. 
Apparently, the model is looking at all the information and is finding more data clumps. But also potentially leading to more false positives.
The temperature also has a similar trade-off. Higher temperatures lead to a lower sensitivity and the other way around.\\
If you prompt a GPT model to refactor source code while also giving it the location of the data clumps, GPT-3.5 and GPT-4's median is identical, lying at 68\%.
These results show if you know where to look for data clumps and which places to refactor, both models can refactor the source code just as well as the other.
GPT-3.5-Turbos arithmetic mean is less which can be explained by the existence of more overall compiler errors.
On the same note: the median of the three instruction variants is also identical.
Higher temperature values also resulted in a median of 0\%, indicating more non-compilable code.\\
The final step of the experiment is combining detection and refactoring into one step. At this point, the limitations of GPT-3.5-Turbo become clear. The model scores a median score of 7\% compared to 82\% of GPT-4.
Surprisingly, however, providing no definitions about data clumps leads to the best results, reaching a median of 46\%. 
All other instruction types are 0\% each.
Another experiment, held in the paper mentioned in section \ref{introduction}, \cite[The effectiveness of supervised machine learning algorithms in predicting software refactoring]{aniche2020effectiveness}, compared different machine learning models with each other, with the "Random Forest" Model performing best out of all the tested models.\\
It appears as if machine learning models commonly perform best in a close-to-random environment.
\subsubsection{Github co-pilot} \cite{imai2022github}
\subsubsection{Fauxpilot Client}
\subsection{Dedicated Models}
\subsubsection{DNNFFz}
\section{Benefits of AI-Powered Refactoring}\label{benefits}
Logically, automated refactoring comes with all the benefits dicussed in the prior chapter, namely more readable, higher quality code that is more maintainable, but more importantly it makes all this possible in a shorter amount of time. \cite{negara2012using}
Automating the refactoring process is not a new idea at all and numerous tools have been developed and modern IDE's already support some refactorings. However those are often small scale and need input on what to refactor. \cite{usedisuserefactor}
It could also be verified that autmated refactorings bring a statistically significant time improvement to nearly every type of refactoring. \cite{negara2013comparative} This of course means less cost in extension. The same study also found that currently automated refactoring is more often used for smaller chnages, since there the refactoring is less error prone. 
Using AI the possible usecases could be significantly expanded and lead to greater time savings.

\section{Challenges and Limitations}
\subsection{Over-reliance on Automation}
Especially when using applications like Chat GPT, users often rely too much on the model to understand their problem and give insufficient and vague requests, which have much room for interpretation. This can then, unsurprisingly, lead to undesired results \cite{alomar2024refactor}.

\subsection{Potential for Unintended Consequences}
Software developers often choose not to use automated refactoring because of the unpredictability of the result. 
"[…] If I cannot guess, I don't use the refactoring. I consider it not worth the trouble. […]" \cite{usedisuserefactor} Since refactoring can span multiple files, it can be challenging for a developer to check all the changes after an automatic refactoring has occurred.
Especially generative AI solutions, which handle smell detection and the generation of solutions, have often been found to provide mixed results, which can lead to the code's behaviour being changed. This also leads to reduced trust in those tools by developers \cite{10.1145/3397481.3450656}.

\subsection{Performance Concerns}
While significant technological strides are being made, it is still a present issue that some automatic refactorings produce errors, significantly when misused \cite{7833023}. 
Especially when using AI-based tools, it is only sometimes given that the code's behaviour is the same as before, making the refactoring itself faulty. Baqais and Alshayeb could find quite a lack of checking for behaviour in many research papers, leaving out a crucial step \cite{baqais2020automatic}.

\subsection{Trust and knowledge gaps}
Some developers stated that when using an AI refactoring tool, it would probably be helpful to know how it works in the background to get better results potentially \cite{10.1145/3397481.3450656}.
This would then require more focus on learning those Models. Also, in another study, Developers did not use specific automated refactorings of IDEs because they needed to learn of their existence \cite{negara2013comparative}.
Education about existing tools alone could help improve performance.


\section{Future Directions and Research Opportunites}
Due to these discussed limitations, there are two main takeaways for problems that must be tackled for AI-powered refactoring to be truly industry-changing.
Firstly there has to be more work put into the systems to avoid behavioural changes of the refactored software. 
Secondly, it will be very important to improve the communication between tools and developers as well as improve education about existing tools so that they can be used to their full potential.

\section{Conclusion and Outlook}
Automated code refactoring is not a new idea and shows accurate results. AI-assisted refactoring brings discoveries on when and, more importantly, why code has to be refactored.
Good tools have already been developed to solve separate stages of refactoring. Yet, fully automatic code refactoring remains challenging.

%\nocite{*} % Makes every ref appear despite not being cited
\bibliographystyle{plain} % We choose the "plain" reference style
\bibliography{refs} % Entries are in the refs.bib file
\end{document}
