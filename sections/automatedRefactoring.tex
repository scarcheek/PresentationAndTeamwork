Logically, automated refactoring comes with all the benefits dicussed in the prior chapter, namely more readable, higher quality code that is more maintainable, but more importantly it makes all this possible in a shorter amount of time. \cite{negara2012using}
Automating the refactoring process is not a new idea at all and numerous tools have been developed and modern IDE's already support some refactorings. However those are often small scale and need input on what to refactor. \cite{usedisuserefactor}
It could also be verified that autmated refactorings bring a statistically significant time improvement to nearly every type of refactoring. \cite{negara2013comparative} This of course means less cost in extension. The same study also found that currently automated refactoring is more often used for smaller chnages, since there the refactoring is less error prone. 
Using AI the possible usecases could be significantly expanded and lead to greater time savings.