Logically, automated refactoring comes with all the benefits discussed in the prior chapter \ref{refactoring}, namely more readable, higher quality code that is more maintainable, but more importantly, it makes all this possible in a shorter amount of time.
Automating the refactoring process is a familiar concept, with the development of numerous tools and support for some refactorings in modern IDEs. However, those are often small-scale and need input on what to refactor \cite{usedisuserefactor}.
Automated refactorings can significantly improve the time taken for nearly every type of refactoring, resulting in less cost in extension\cite{negara2013comparative}. The same study also found that automated refactoring is more often used for more minor changes since the refactoring is less error-prone. 
AI could significantly expand the possible use cases for automated code refactoring, resulting in more major time savings.