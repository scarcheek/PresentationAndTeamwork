Logically, automated refactoring comes with all the benefits discussed in the prior chapter \ref{refactoring}, namely more readable, higher quality code that is more maintainable, but more importantly, it makes all this possible in a shorter amount of time. \cite{negara2012using}
Automating the refactoring process is not a new idea at all numerous tools have been developed and modern IDEs already support some refactorings. However, those are often small-scale and need input on what to refactor. \cite{usedisuserefactor}
It could also be verified that automated refactorings bring a statistically significant time improvement to nearly every type of refactoring. \cite{negara2013comparative} This of course means less cost in extension. The same study also found that currently automated refactoring is more often used for smaller changes since the refactoring is less error-prone. 
Using AI the possible use cases could be significantly expanded and lead to greater time savings.