\subsection{Over-reliance on Automation}
Especially when using applications like Chat GPT, users often rely too much on the model to understand their problem and give insufficient and vague requests, which have much room for interpretation. This can then, unsurprisingly, lead to undesired results \cite{alomar2024refactor}.

\subsection{Potential for Unintended Consequences}
Software developers often choose not to use automated refactoring because of the unpredictability of the result. 
"[…] If I cannot guess, I don't use the refactoring. I consider it not worth the trouble. […]" \cite{usedisuserefactor} Since refactoring can span multiple files, it can be challenging for a developer to check all the changes after an automatic refactoring has occurred.
Especially generative AI solutions, which handle smell detection and the generation of solutions, have often been found to provide mixed results, which can lead to the code's behaviour being changed. This also leads to reduced trust in those tools by developers \cite{10.1145/3397481.3450656}.

\subsection{Performance Concerns}
While significant technological strides are being made, it is still a present issue that some automatic refactorings produce errors, significantly when misused \cite{7833023}. 
Especially when using AI-based tools, it is only sometimes given that the code's behaviour is the same as before, making the refactoring itself faulty. Baqais and Alshayeb could find quite a lack of checking for behaviour in many research papers, leaving out a crucial step \cite{baqais2020automatic}.

\subsection{Trust and knowledge gaps}
Some developers stated that when using an AI refactoring tool, it would probably be helpful to know how it works in the background to get better results potentially \cite{10.1145/3397481.3450656}.
This would then require more focus on learning those Models. Also, in another study, Developers did not use specific automated refactorings of IDEs because they needed to learn of their existence \cite{negara2013comparative}.
Education about existing tools alone could help improve performance.
