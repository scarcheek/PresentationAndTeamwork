\subsection{Over-reliance on Automation}
Especially when using applications like Chat GPT users often rely too much on the model to understand their problem and give insufficient and/or vague requests which have a lot of room for interpretation. This can then, unsurprisingly, lead to undesired results \cite{alomar2024refactor}.

\subsection{Potential for Unintended Consequences}
A big reason software developers often choose to not use automated refactoring is the unpredictability of the result. 
"[…] If I cannot guess, I don't use the refactoring. I consider it not worth the trouble. […]" \cite{usedisuserefactor} Since refactoring can span multiple files it can be hard for a developer to check all the changes after an automatic refactoring has occurred.
Especially generative AI solutions which do not only handle smell detection but also the generation of solutions have often been found to provide mixed results, which can lead to the behaviour of the code being changed. This also leads to reduced trust in those tools by developers \cite{10.1145/3397481.3450656}.

\subsection{Performance Concerns}
While big strides in the technology are being made, it is still a present issue that some automatic refactorings do produce errors especially when misused \cite{7833023}. 
Especially when using AI-based tools it is not always given that the behaviour of the code is indeed the same as before, making the refactoring itself faulty. Baqais and Alshayeb could find quite a lack of checking for behaviour in a multitude of research papers, leaving out a crucial step \cite{baqais2020automatic}.

\subsection{Trust and knowledge gaps}
Some developers stated, that when using an AI refactoring tool, it would probably be useful to know how it works in the background to potentially get better results \cite{10.1145/3397481.3450656}.
This would then require more focus on learning those Models. Also in another study Developers did not use certain automated refactorings of IDE's because they did not know of their existence \cite{negara2013comparative}.
In both of those cases better education about already existing tools alone could help improve performance.
