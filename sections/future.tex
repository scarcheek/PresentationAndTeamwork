Due to the discussed limitations in section \ref{challenges}, we can derive two takeaways.
We must continue to conduct more research to determine the accuracy of the refactored code produced by AI systems. This is not just a task for a few but a collective effort that requires the expertise and dedication of all involved in AI-assisted code refactoring.
Moreover, it is within our reach to enhance the communication between tools and developers and to better educate about existing tools. This will empower software engineers to fully exploit these tools, fostering a more efficient and effective code refactoring process.
Automated code refactoring is a familiar idea and shows accurate results. AI-assisted refactoring brings discoveries on when and, more importantly, why code has to be refactored. GPT models can already be used effectively, properly refactoring found research opportunities. Random Forest models are great at discovering refactoring opportunities, especially when trained on specific datasets. As fully automated code refactoring is still faulty, more research has to be conducted to increase trustability and allow reliance on AI-assisted code refactoring.