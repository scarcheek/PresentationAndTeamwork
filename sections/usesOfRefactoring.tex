Fowler states refactoring describes changing the internal structure of code without changing its behaviour.\cite{fowler2018refactoring} 
Refactoring can lead to improvements in code in various aspects. By reducing the code's complexity while keeping its behaviour identical, it becomes more human-readable and, therefore, more maintainable since it is easier to expand on it in the future. \cite{kaur2016analysis}
Another great advantage of refactoring is the reduction of technical debt. This term refers to a metaphorical debt when subpar solutions get chosen to accelerate software production. While the chosen solutions might suffice they often do not take further developments into account and cause problems later. \cite{techdebt}
And of course, the overall code quality can be improved.
The main difficulty in code refactoring lies in finding places to optimize, otherwise called "bad smells". These bad smells can be found virtually anywhere in the source code. Refactoring can happen on different levels, like class-level refactoring describing the extraction of classes into one or more sub-classes, while variable-level refactoring could be something like executing an action inline instead of creating an unnecessary variable. \cite{aniche2020effectiveness} This makes it very difficult to locate these bad smells and as a consequence time consuming and expensive.