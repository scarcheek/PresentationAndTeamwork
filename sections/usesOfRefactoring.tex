Fowler states refactoring describes changing the internal structure of code without changing its behaviour.\cite{fowler2018refactoring} 
This can lead to improvements in code in various aspects. By reducing the complexity of code while keeping its behavior intact it becomes more human-readable and therefore also more maintainable since it is easier to expand on it in the future. \cite{kaur2016analysis}
Another great advantage of refactoring is the reduction of technical debt. This term refers to a metaphorical debt when subpar solutions have been chosen to accelerate the production of software. While the chosen solutions might work they often do not take into account further developments and through this often cause problems in the future. \cite{techdebt}
And of course, the overall code quality can be improved.
The main difficulty in code refactoring lies in finding places to optimize, otherwise called "bad smells". These bad smells can and therefore refactoring can happen on different levels within the code, like Class level refactoring describing for example the extraction of a class into one or more sub-classes, while a variable level refactoring could be something like executing an action inline instead of creating an unnecessary variable. \cite{aniche2020effectiveness} This makes it very difficult to locate these bad smells and as a consequence time consuming and expensive.